%----------------------------------------------------------------------------------------
%    PAGE ADJUSTMENTS
%----------------------------------------------------------------------------------------

\documentclass[12pt,a4paper]{article}                      % Article 12pt font for a4 paper while hiding links
\usepackage[margin=1in]{geometry}                          % Required to adjust margins
\usepackage{booktabs}

%----------------------------------------------------------------------------------------
%    TYPE SETTING PACKAGES
%----------------------------------------------------------------------------------------

\usepackage[english]{babel}                                % English language/hyphenation 
\usepackage[utf8x]{inputenc}                               % Accept different input encodings
\usepackage{amsmath,amsfonts,amsthm,amssymb}               % Math packages to use equations
\usepackage{siunitx}                                       % Scientific units and numbering
\usepackage[usenames,dvipsnames,svgnames,table]{xcolor}    % Set color of text/background
\linespread{1.2}                                           % Default line spacing size
\usepackage{microtype}                                     % Improves spacing in the document
\usepackage{setspace}                                      % Set line spacing dynamically
\usepackage{tocloft}                                       % List adjustments including ToC
\usepackage{pdflscape}
\DeclareMathOperator*{\argmin}{arg\!\min}
%----------------------------------------------------------------------------------------
%    FIGURES
%----------------------------------------------------------------------------------------

\usepackage{graphicx}                                      % Required for the inclusion of images
\graphicspath{{./resources/}}                              % Specifies picture directory
\usepackage{float}                                         % Allows putting an [H] in \begin{figure}
\usepackage{wrapfig}                                       % Allows in-line images

\usepackage[colorlinks=true, citecolor=blue]{hyperref}     % References
\usepackage{cleveref}                                      % Better References
%\crefname{lstlisting}{listing}{listings}
%\Crefname{lstlisting}{Listing}{Listings}
\crefname{figure}{figure}{figures}
\Crefname{figure}{Figure}{Figures}

%----------------------------------------------------------------------------------------
%    INCLUDE CODE
%----------------------------------------------------------------------------------------

\usepackage{listings}                                      % Package so code looks pretty
\lstset{
language=Python,                                           % Choose the language
basicstyle=\footnotesize,                                  % The size of the fonts used
numbers=left,                                              % Where to put the line-numbers
numberstyle=\footnotesize,                                 % The size of the line-numbers
stepnumber=1,                                              % The step line-numbers
numbersep=5pt,                                             % How far the line-numbers are from the code
backgroundcolor=\color{white},                             % Choose the background color
showspaces=false,                                          % Show spaces adding partiular underscores
showstringspaces=false,                                    % Underline spaces within strings
showtabs=false,                                            % Show tabs within strings adding particular underscores
frame=single,                                              % Adds a frame around the code
tabsize=2,                                                 % Sets default tabsize to 2 spaces
captionpos=b,                                              % Sets the caption-position to bottom
breaklines=true,                                           % Sets automatic line breaking
breakatwhitespace=false,                                   % Sets if automatic breaks should only happen at whitespace
escapeinside={\%*}{*)}                                     % If you want to add a comment within your code
}

\usepackage[T1]{fontenc}
\usepackage{inconsolata}

%----------------------------------------------------------------------------------------
%    EXTRAS
%----------------------------------------------------------------------------------------

\usepackage{attachfile}                                    % Attach files to your document
\usepackage{fancyhdr}                                      % Fancy Header
\usepackage{natbib}

\begin{document}

%----------------------------------------------------------------------------------------
%    COMMANDS
%----------------------------------------------------------------------------------------

\renewcommand*\thesection{\arabic{section}}                % Renew section numbers
\renewcommand{\labelenumi}{\alph{enumi}.}                  % Section ordered numbering
\let\oldvec\vec                                            % Save the old vector style
\renewcommand{\vec}[1]{\oldvec{\mathbf{#1}}}               % Set vectors to look like vectors

\renewcommand{\contentsname}{Table of Contents}            % Make ToC actually say ToC
\addtocontents{toc}{~\hfill\textbf{Page}\par}              % Add 'page' to top of ToC
\renewcommand{\cftsecleader}{\cftdotfill{\cftdotsep}}      % Makes dots leading up to page number
\setcounter{tocdepth}{3}                                   % Depth of ToC
\setcounter{lofdepth}{3}                                   % Depth of LoF

\pagestyle{fancy}                                          % Fancy page style for headers
\setlength{\headheight}{15pt}                              % Change header height
\fancyhead[L,LO]{\fontsize{8}{10} \selectfont \firstmark}  % Adds header to left with section name
\fancyhead[R,RO]{\fontsize{8}{10} \selectfont Right}       % Adds header to right
\definecolor{grey}{HTML}{cccccc}                           % The next 4 lines modifies the header (color)
\renewcommand{\headrulewidth}{1px}
\renewcommand{\headrule}{{\color{grey}%
\hrule width\headwidth height\headrulewidth%
\vskip-\headrulewidth}}

\numberwithin{equation}{section}                           % Number equations within sections
\numberwithin{figure}{section}                             % Number figures within sections
\numberwithin{table}{section}                              % Number tables within sections
\numberwithin{lstlisting}{section}                         % Number listings within sections

\renewcommand{\sfdefault}{phv}                             % Change default font
\renewcommand{\familydefault}{\sfdefault}                  % Use default font everywhere
\newcommand{\tvect}[2]{%
  \ensuremath{\Bigl(\negthinspace\begin{smallmatrix}#1\\#2\end{smallmatrix}\Bigr)}}

%----------------------------------------------------------------------------------------
%    TITLE PAGE
%----------------------------------------------------------------------------------------

\begin{titlepage}

\title{Notes}
\author{Tom Augspurger}                               % via Seth Miers

\vspace*{\fill}                                            % Center title page vertically

% \newcommand{\HRule}{\rule{\linewidth}{0.3mm}}              % Defines horizontal lines

\center                                                    % Center everything on the page

% \textsc{\LARGE First Heading}\[1.5cm]                % First heading
% \textsc{\Large Major Heading}\[0.5cm]                % Major heading
% \textsc{\large Minor Heading}\[0.5cm]                % Minor heading

% \HRule \[0.4cm]
% { \huge \bfseries Title}\[0.4cm]                     % Title of document
% \HRule \[1.5cm]

%\begin{minipage}{0.4\textwidth}
%\begin{flushleft} \large
%\emph{Author:}\
%Tom \textsc{Augspurger}                                    % Name
%\end{flushleft}
%\end{minipage}
%~
%\begin{minipage}{0.4\textwidth}
%\begin{flushright} \large
%\emph{Professor:} \
%Dr. James \textsc{Smith}                                   % Professor's Name
%\end{flushright}
%\end{minipage}\[4cm]

% {\large \today}\[3cm]                                     % Date, change the \today to be precise

%\includegraphics{Logo}\[1cm]                             % Include a department/university logo

\vspace*{\fill}                                            % Fill the rest of the page with whitespace

\end{titlepage}

% \phantomsection

% \tableofcontents                                           % These two lines are needed to
% \addcontentsline{toc}{section}{Table of Contents}          % initialize and display TOC
% \listoffigures                                             % These two lines are needed to
% \addcontentsline{lof}{section}{List of Figures}            % initialize and display LOF
% \newpage

%----------------------------------------------------------------------------------------
%    CONTENT
%----------------------------------------------------------------------------------------

\section{Theoretical Framework}
\label{sec:theoretical_framework}

To estimate the trade elasticities, I follow \cite{broda_weinstein_2006_globalization_gains} who in turn follow \cite{feenstra_1994_new_product_varieties}.  The estimation strategy is complicated by the introduction of new varieties.  In standard love-of-variety type utility functions, failure to account for new varieties is a serious omission.  The outline here will be a very rough sketch.  For a more thorough treatment, either \cite{feenstra_1994_new_product_varieties} or \cite{broda_weinstein_2006_globalization_gains} will suffice.

For our purposes, a variety is defined as a particular good defined by the CN8 nomenclature (see \ref{sec:data_processing}), produced in a particular country.  So, for example, Polystyrene produced in Germany is one variety, while Polystyrene produced in France is another.

\subsection{Feenstra Price Index}
\label{sub:feenstra_price_index}

Here, the goal is to build an aggregate price index that accounts for the increasing varieties available for consumption.  We'll use a three-level utility function that aggregates imported varieties into composite imported goods (one for each good), and a further level that aggregates these composite goods into a single composite imported good.  This composite imported good is then combined with a domestic good in the highest level of the utility function.  That is,

\begin{equation}
    U_t = (D_t^{(\kappa - 1)/\kappa} + M_t^{(\kappa - 1)/\kappa})^{\kappa/(\kappa - 1)}; \quad \kappa > 1
\end{equation}

where $M_t$ is the composite imported good and $D_t$ is the domestic good and $\kappa$ the elasticity of substitution between them.  For the next tier,

\begin{equation}
    M_t = \Big(\sum_{g \in G} M_{gt}^{(\gamma - 1) / \gamma}\Big)^{\gamma/(\gamma - 1)}; \quad \gamma > 1
\end{equation}

where $M_gt$ is the subutility from consumption of each imported good $g$ at time $t$, and $\gamma$ is the elasticity of substitution among them.  $G$ is the set of all imported goods.

We give $M_gt$ the form

\begin{equation}
    M_gt = \Big( \sum_{c \in C} d_{gct}^{1 / \sigma_g}(m_{gct})^{(\sigma - 1) / \sigma_g}  \Big) ^ {\sigma / (\sigma_g - 1)};`' \quad \sigma_g > 1 \ \ \forall g \in G
\end{equation}

where $\sigma_g$ is the elasticity of substitution among varieties of good $g$, assumed to be greater than one.  The subscript $c$ refers to the country of origin for good $g$.  $C$ is the set of all countries and $d_{gct}$ is a taste parameter for goods of type $g$ imported from country $c$.

Given this specification, we can then derive the form of the minimum cost bundle needed to achieve a given utility level.  I skip the derivation here and note that the results are a vector $\mathbf{p}_{gt}$ containing elements $p_{gct}$, the prices of individual varieties, and another vector $\mathbf{x}_{gt}$ containing elements $x_{gct}$, the cost-minimizing quantities for each variety.  From this we get the cost shares (which are used in estimation):

\begin{equation}
    s_{gct} = \frac{p_{gct}x_{gct}}{\sum_{c \in I_g}p_{gct}x_{gct}}
\end{equation}

where $I_g$ is an index of varieties consumed at both times $t$ and $t - 1$.

\section{Estimation Strategy}
\label{sec:econometric_methodology}

We have in hand the cost shares $s_{gct}$ and prices $p_{gct}$. This lets us express the import demand for each variety (differenced over time) as

\begin{equation}\label{eq:import_demand}
    \Delta \ln{s_{gct}} = \varphi_{gt} - (\sigma_g - 1) \Delta \ln{p_gct} + \varepsilon_{gct}
\end{equation}

where $\varphi_{gt} = (\sigma_g - 1) \ln{[\phi_{gt}^M (d_t) / \phi_{gt - 1}^M (d_{t - 1})]}$ is a random effect since $d_t$ is random and $\varepsilon_{gct} = \Delta \ln{d_{gct}}$.

Likewise, we can write the export supply equation as:

\begin{equation}\label{eq:export_supply}
    \Delta \ln{p_{gct}} = \psi_{gt} + \frac{\omega_g}{1 + \omega_g} \Delta \ln{s_{gct} + \delta_{gct}}
\end{equation}

where $\psi = -\omega_g \Delta \ln{E_gt / (1 + \omega_g)}$, $\omega_g$ is the inverse supply elasticity (constant across countries by assupmtion) and $\delta_t =  \Delta \ln{v_{gct} / (1 + \omega_g)}$ picks up random changes from a technology factor $v_{gct}$.  We assume that $E(\varepsilon_{gct}\delta{gct} = 0$, that is demand and supply errors at the variety level are uncorrelated once good-time specific effects are accounted for.

To aid in estimation, we wish to eliminate $\varphi_{gt}$ and $\psi_{gt}$.  To accomplish this, for each good $g$ choose a reference country $k$ which supplied the good each period.  Then difference the demand and supply equations \ref{eq:import_demand} and \ref{eq:export_supply} relative to country $k$ to get:

\begin{equation}\label{eq:diffed_import_demand}
    \Delta^k \ln{s_{gct}} = - (\sigma_g - 1) \Delta^k \ln{p_gct} + \varepsilon^k_{gct}
\end{equation}

\begin{equation}\label{eq:diffed_export_supply}
    \Delta^k \ln{p_{gct}} = \frac{\omega_g}{1 + \omega_g} \Delta^k \ln{s_{gct}} + \Delta^k_{gct}
\end{equation}

where $\Delta^k x_{gct} \equiv \Delta x_{gct} - \Delta x_{gkt}$, $\varepsilon^k_{gct} \equiv \varepsilon_{gct} - \varepsilon_{gkt}$ and $\delta^k_{gct} \equiv \delta_{gct} - \delta_{gkt}$.
Using $E(\varepsilon_{gct}^k \delta^k_{gct} = 0)$, multiply \ref{eq:diffed_import_demand} and \ref{eq:diffed_export_supply} to get:

\begin{equation}
    (\Delta^k \ln_{pgct})^2 = \theta_1 (\Delta^k \ln s_{gct})^2 + \theta_2 (\Delta^k \ln{p_{gct}} \Delta^k\ln{s_{gct}}) + u_{gct}
\end{equation}

where

\begin{equation}\label{eq:multiplied}
    \theta_1 = \frac{\omega_g}{(1 + \omega_g) (\sigma_g - 1)}, \quad \theta_2 = \frac{1 - \omega_g(\sigma_g - 2)}{(1 + \omega_g)(\sigma_g - 1)}
\end{equation}

and

\begin{equation}
    u_{gct} = \varepsilon^k_{gct} \delta^k_{gct}.
\end{equation}

\subsection{Econometric Methodology}
\label{sub:econometric_methodology}


Our parameters of interest $\beta_g \equiv \tvect{\sigma_g}{\omega_g}$ cannot be estimated consistently from \ref{eq:multiplied} since $u_{gct}$ is correlated with the regressands.  However, we can achieve consistency from our assumption that the demand and supply elasticities are constant over varieties of the same good.  With this assumption we get a set of moment conditions for each good $g$:

\begin{equation}
    G(\beta_g) = E_t(u_{gct}(\beta_g)) = 0 \quad \forall \ c
\end{equation}

This equation can be estimated using GMM:

\begin{equation}
    \hat{\beta}_g = \argmin_{\beta \in B} G^*(\beta_g)^{T} W G^*(\beta_g)
\end{equation}

where $G^*(\beta)$ is the sample equivalent of $G(\beta)$, $W$ is a positive definite weighting matrix, and $B$ is the set of feasible $\beta$, defined by economic theory to be $\sigma_g > 1, \omega_g > 0$.

The choice of weighting matrix is motivated by pragmatism.  There's good reason to worry that unit values calculated on small volumes are less precisely measured than those based on large volumes.  For this reasons we choose $W$ such that:

\begin{equation}
    W \equiv T^{3/2}\Big(\frac{1}{q_{gct}} + \frac{1}{q_{gct - 1}})^{-1/2}
\end{equation}

where $T$ is the number of periods in our sample (12 in my case) and $q_{gct}$ is the quantity imported at time $t$, which is known from data.  \cite{broda_weinstein_2006_globalization_gains} also add a term

\begin{equation}
    \hat{\chi}^2 \frac{1}{T} \sum_t \big(\frac{1}{q_{gct}} + \frac{1}{q_{gct-1}})
\end{equation}

to the right-hand side of \ref{eq:multiplied}.  Together these should account for our measurement concerns.  See the appendix of \cite{broda_weinstein_2006_globalization_gains} for details.

\section{Estimation Results}
\label{sec:estimation_results}

Presenting these results proves to be something of a challenge.  My sample includes 27 countries and one aggregate for the EU (see table \ref{tab:declarants}).  

At this step it became clear that the data likely had some massive outliers (or our manipulations had introduced some).
This probably shouldn't be too surprising.
I'm still analyzing the chain from raw data to elasticity estimate, but anecdotally some things stand out:

For example, France's largest elasticity was for good number 72122011.  The estimated elasticity was 3963.  But this was based off just 12 observations, 5 countries for 2 years and 2 countries for a third year.
It might be worthwhile to exclude those goods with relatively small samples and rerun the analysis.

The next two tables present some summary statistics for the import elasticities for each of the 28 geographic regions in our sample.  The first contains all the valid estimates, while the second excludes those outside of three standard deviations from the mean.\\

\noindent
\input{summary_stats.txt}\label{tab:summary_stats}
\newpage

\noindent
\input{summary_stats_inliers.txt}\label{tab:inlier_summary_stats}

\newpage

The next two figures are an attempt to visualize the results.
Recall that our estimation focused on two parameters, the import demand elasticity and the export supply elasticity: $\tvect{\sigma_g}{\omega_g}$.
The next graphs plot the results in $\tvect{\sigma_g}{\omega_g}$-space, so our main interest is on the horizontal axis.
Like above the first is with all valid estimates, while the second removes outliers.

It's not immediately clear what pattern we should expect for these goods.
I grouped the results by the two-digit level and no clear pattern emerged, so I do not report those results here.

\newgeometry{margin=1cm}
\begin{landscape}
\thispagestyle{empty}
\begin{center}
\footnotesize 9
    \includegraphics[scale=.75]{scatter.png}
\end{center}
\end{landscape}
\restoregeometry


\newgeometry{margin=1cm}
\begin{landscape}
\thispagestyle{empty}
\begin{center}
\footnotesize 9
    \includegraphics[scale=.75]{inliers_scatter.png}
\end{center}
\end{landscape}
\restoregeometry

In the next figure, I plot histograms of the elasticity estimates, again with the outliers removed.

\newgeometry{margin=1cm}
\begin{landscape}
\thispagestyle{empty}
\begin{center}
\footnotesize 9
    \includegraphics[scale=.75]{histograms.png}
\end{center}
\end{landscape}
\restoregeometry

On average, the higher the level of aggregation, the lower the estimate of elasticity.
In this table I aggregated to the two-digit CN level and estimated the elasticity of import demand in the same way.
The average change from before (disaggregated) to now (aggregated) is negative.\\

\noindent
\input{level_two_means.txt}\label{tab:two_means}


%----------------------------------------------------------------------------------------
\bibliographystyle{apa}
\bibliography{citations}

\newpage

\section{Appendix}
\label{sec:appendix}





\section{Data Processing}
\label{sec:data_processing}

The raw data can be accessed via \href{http://epp.eurostat.ec.europa.eu/NavTree\_prod/everybody/BulkDownloadListing?sort=1\&dir=comext%2F2012S2%2Fdata}{Eurostat}.  This page contains links to files annual COMEXT trade data from 1988 to 2011.  I describe the data further elsewhere.


Once downloaded, the files can be processed using three \href{www.python.org}{Python} scripts: \texttt{download\_to\_hdf5.py}, \texttt{differencing.py}, and \texttt{generate\_prices\_shares.py}.

 
The first script, \texttt{download\_to\_hdf5.py}, takes the gziped files from the Comext Database (one for each year), extracts them using the \texttt{7z} extractor, and writes them to an \texttt{hdf5} datafile.  Some intermediate logic is applied obtain a unique index, split the data by declarant, and write each declarant separately.


Next, I calculate the cost shares and prices described in \ref{sec:econometric_methodology}.  The code here is self-explanatory. Note that goods with no values for \texttt{Quantity} were dropped. To save on space I overwrites the database generated by \texttt{download\_to\_hdf5.py}.  This can be avoided by changing the filename written to.


Finally, I use the script \texttt{differencing.py} which handles finding a reference country who exported the product each year for each good, and then differencing the prices and cost shares by year and with that reference country's values.  Note that some goods had to be dropped because they did not contain a valid reference country.

\section{Data Description}

The bulk of my data is from the Comext database hosted on \href{http://epp.eurostat.ec.europa.eu/portal/page/portal/eurostat/home/}{Eurostat}.  I used data from the years 2000-2011.  Here is a sample:

  \begin{table}[h]\label{tab:sample_row}
  \begin{tabular}{|c|c|c|c|c|c|c|c|c|}\footnotesize
    & & & & & Value (1000 Euro) & Quantity (Ton) & Sup Q \\ \hline
    Flow & Period & Dec. & Product & Partner &  &  &     \\ \hline
    1 & 2009-01-01 & 1  & 01011010 & 3       & 3741.08 & 1359.2 & 0\\ \hline
  \end{tabular}
  \end{table}

\texttt{Flow} refers to the record being an import (1) or export (2).  \texttt{Period} is the year that the record refers to. \texttt{Declarant} is the code for the country doing the importing, France in this case.  \texttt{Product} refers to the 8-digit \hyperref{http://ec.europa.eu/eurostat/ramon/nomenclatures/index.cfm?TargetUrl=LST_NOM_DTL&StrNom=CN_2013&StrLanguageCode=EN&IntPcKey=&StrLayoutCode=HIERARCHIC}{Combined Nomenclature} classification system.  Finally, \texttt{Partner} refers to the country of origin for this record.  This forms a unique index.  The data used for estimation are \texttt{Value} (in thousands of Euros) and \texttt{Quantity} (in tons).  \texttt{Sup Q} is a supplemental quantity value.  There are some items with all zeros in \texttt{Quantity} and nonzero values in \texttt{Sup Q}.  As it stands, these values were dropped from my calculation while in principal they could have been included.  The difficulty comes from some items being measured by \texttt{Quantity} some years and \texttt{Sup Q} in other years.

From these data, we can calculate the cost shares and prices described in \ref{sec:theoretical_framework}, using the methods described in \ref{sec:econometric_methodology}.

\section{Estimation Described}
\label{sec:estimation_described}

The bulk of the work is done by the script \texttt{gmm\_with\_weighting.py}.  For each declarant in our sample, I split the declarant's import data by good, apply the GMM estimation function, and combine the results into a table containing our two parameters of interest, $\tvect{\sigma_g}{\omega_g}$, for each good.

As the filename suggests, the weighting matrix is also calculated here, as described in \ref{sec:econometric_methodology}.  This involves differencing the quantity data by year.  \cite{broda_weinstein_2006_globalization_gains} do not comment on what should happen to the values from the first observation, so I fill these in the the mean from the other years.

\end{document}


