\documentclass[11pt]{article}
\usepackage{graphicx}
\usepackage{amssymb}
\usepackage{amsmath}
\usepackage{natbib}
\graphicspath{{/Users/tom/TradeData/data-wrangling/paper/figures/}}

\title{European Trade and the Great Recession}
\author{Tom Augspurger}
\date{}
\begin{document}
\maketitle
\begin{abstract}
  Trade collapse.
\end{abstract}
\section{Introduction}
\label{sec:introduction}

\section{Background Literature}
\label{sec:background_literature}
  The Great Recession sparked a mini boom in research on trade collapses.  Several papers are relevant here.

  Several notable hypothesis have been offered to explain the large decline in trade during the Great Recession.  The first, discussed in Chor and Mannova (2012) \cite{chor-manova:2012}, posits financing was an important cause of the decline in trade.

\section{Stylized Facts} % (fold)
\label{sec:stylized_facts}



\section{Trade Wedges} % (fold)
\label{sec:trade_wedges}
  Has the decline in trade been abnormally large?  To get a rough answer to this question, we can calculate what Levchnko \emph{et. al.} \cite{llt:2010} call a ``Trade Wedge''.  The ``wedge'' refers to the gap between the observed decline in trade and what would be predicted by a standard model.  The model specifies an import demand relationship as:
  \begin{equation}
      \hat{y}^f = \varepsilon(\hat{P} - \hat{p}^f) + (\hat{C + I})
  \end{equation}
  where carets indicate log changes, $y$ is the demand for imports, $P$ is the overall price level, $p^f$ is the price level of imports, and $(C + I)$ is consumption plus investment (aggregate demand).  Finally $\varepsilon$ is the elasticity of substitution.  If our simple model is entirely correct, this equation should hold exactly.  The difference between the calculated and predicted values is the trade wedge.  Since we're calculating log changes, $\hat{P}$ and $\hat{p}^f$ are log changes in the deflator for GDP and imports.

  \begin{figure} \label{fig:trade_wedge}
    \centering
      \includegraphics[scale=.33]{trade_wedge_1and5.png}
    \caption{Trade Wedges}
  \end{figure}

  As shown in \ref{fig:trade_wedge}, every country in this sample experienced a larger decline in trade than predicted by the model.  Of course, this may say more about our model that about whether the decline in trade was particularly large.  But, as a first-order approximation, it suggests that the decline in trade may be worth looking into further.

\section{Hypothesis}
\label{sec:hypothesis}
  Unfortunately, I had neither the data nor the time to examine the trade credit hypothesis.  This also means that I had neither the data nor the time to control for the effects of trade credit.  So to the extent that trade credit is an important player in our story my model will be misspecified.

\section{Empirical Results} % (fold)
\label{sec:empirical_results}
  Most everything in this section should be tagged with seven levels of asterisks noting the wild misspecification of my model.  But here goes.

  
\section{Conclusion}
\label{sec:conclusion}

\section{Data Appendix}
\label{sec:data_appendix}

\section{Calculation Appendix}
\label{sec:calculation_appendix}



\bibliographystyle{plain}
\bibliography{citations}

\end{document}